% !TEX encoding = UTF-8 Unicode

\linespread{1.7}
\chapter{0-5 Hz Deterministic 3D Ground Motion Simulations for the 2014 $M_w$ 5.1 LaHabra Earthquake}
\linespread{2.0}
%\newrefsection
\label{chap:highf}

\graphicspath{{/Users/zhh076/work/PhD_way/high_f/}}

We have simulated 0-5 Hz deterministic simulations for the 2014 $M_w$ 5.1 La Habra, CA, earthquake in a mesh from the Southern California Earthquake Center (SCEC) Community Velocity Model CVM-S4.26-M01 and a finite-fault source. We include statistical distributions of small-scale crustal heterogeneities (SSHs) with an anisotropic von Karman autocorrelation to represent the effects of realistic velocity and density perturbations, frequency-dependent attenuation $Q(f)$, $V_{S30}$-based near-surface velocity calibration and surface topography using a curvilinear grid \citep{oreillyHighorderFiniteDifference2021}. Strong motion data at 259 sites within an 148 by 140 km area are used to validate our simulations. Our results show that the effects of topography, SSHs and $Q(f)$ become increasingly important and should be included as frequencies increase. Particularly, topography and SSHs predominantly decrease the peak velocities and significantly increase the durations, providing an improved fit to the data. Frequency-dependent attenuation also improves the fit to the data, including the high frequency decay in the Fourier amplitude spectra. The fit between data and synthetics for below 3 Hz was significantly improved by updating the seismic velocities in the top 1000 m of the CVM using the \citet{thompsonUpdatedVs30Map2018} $V_S$ map. The results provide guidance on the importance of the different features controlling high-frequency seismic hazard analysis.


%%%%%%%%%%%%%%%%%%%%%%%%%%
\section{Introduction} \label{highf:intro}
It is the ultimate goal for ground motion modelers to deliver their results to engineers and see their work used in real applications, such as structural design. This is particularly useful in cases of infrequent observations, such as large magnitude events at small distances from the fault, where simulations may provide a viable alternative to data. Deterministic ground motion predictions, including features such as 3D velocity structure and frequency-independent anelastic attenuation are now routinely produced for frequencies up to about 1 Hz with generally satisfactory fit to recorded data \citeg{olsen2009shakeout,roten3DSimulationsEarthquakes2012}. These simulations have been introduced in real applications and proved to be useful, for example in public earthquake emergency response and seismic hazard management \citep{gravesBroadbandGroundMotionSimulation2010}, and complementing the empirical ground motion prediction models in regions with sparse stations coverage \citep{dayModelBasinEffects2008}.  However, structural engineers need ground motions with signal content up to 5 Hz and higher for design purposes. Hybrid techniques, combining deterministic low-frequency and stochastic high-frequency signals \citeg{olsen2015sdsu,gravesKinematicGroundMotion2016} can be used to generate synthetic seismograms with frequency content up to 10 Hz and higher. However, simulating the higher frequency content of the predicted ground motions using a deterministic approach has the potential to lower part of the epistemic uncertainty present in the hybrid methods. In this study we investigate the feasibility of increasing the highest frequency for the deterministic ground motion predictions to 5 Hz, using simulations and data for the 2014 $M_w$ 5.1 La Habra, CA, earthquake. The La Habra event was chosen due to an abundance of records available in the Los Angeles area, while ground motions can be considered linear due to its limited magnitude.


As frequencies increase above about 1 Hz, a variety of features become important to realistically predict deterministic ground motions. For example, small-scale complexity of both the source and surrounding media, on the order of 10-100s of meters, increasingly affects the ground motion predictions at higher frequencies. Frequency-independent anelastic attenuation, often chosen as proportional to the local velocity structure \citeg{gravesBroadbandGroundMotionSimulation2010,savranGroundMotionSimulation2019}, is usually a good approximation for lower frequencies (e.g., up to ~1 Hz). However, models of frequency-independent anelastic attenuation appear to be inconsistent at higher frequencies where regional studies indicate that larger Q values may be more accurate \citeg{withersMemoryEfficientSimulation2015}. Finally, ground motion simulations often artificially truncate the lowest near-surface velocities due to computational limitations, which may be a reasonable approximation for lower frequencies \citeg{olsenEstimationLongPeriodSec2003}. However, stronger effects from this near-surface material emerge as frequencies increase and wavelengths decrease \citeg{pitarka2009simulating,imperatoriBroadbandNearfieldGround2013}. Here, we quantify the effects of all of these features in our 3D simulations of the La Habra event.

In southern California, two 3D state-of-the-art velocity models, namely the Community Velocity Models (CVM) versions S and H have been developed through the Southern California Earthquake Center (SCEC). These CVMs have been validated against ground motion data in a series of studies \citeg{tabordaEvaluationSouthernCalifornia2016,savranGroundMotionSimulation2019,laiShallowBasinStructure2020}, where conclusions often point to inaccuracies in the velocity structure, in particular for the near surface material. \citet{elyVs30derivedNearsurfaceSeismic2010} proposed a method to calibrate the near-surface material based on estimates of the time-averaged velocity in the upper 30 m ($V_{S30}$), and later improved by \citet{huCalibrationNearsurfaceSeismic2021}, specifically for rock sites. In this study, we use the SCEC CVM-S4.26-M01 with the near-surface $V_S$ tapering method proposed by \citet{huCalibrationNearsurfaceSeismic2021}.

The effects of irregular surface topography on ground motions also play an increasingly large role as frequencies increase \citeg{liuScatteringSeismicWaves2020}. In addition, in recent studies, theoretical and numerical methods have helped clarify the interaction between seismic waves and topography \citep[mainly scattering and trapping of waves, e.g.,][]{imperatoriRoleTopographyLateral2015,takemura2015scattering}, as well as describing the characteristic effects on ground motions. Some of the most notable effects of topography include: (1) Amplification tends to occur at the top of relatively steep slopes for waves with comparable wavelength to the size of the topographic features; on the other hand, deamplification tends to occur at low-elevation areas \citep{trifunacAnalysisPacoimaDam1971,booreNoteEffectSimple1972,spudichDirectionalTopographicSite1996,bouchonSeismicResponseHill1996,assimakiSoilDependentTopographicEffects2005}. Amplification can range up to a factor of 10 or more between the crest and base of a topographic feature \citep{davisObservedEffectsTopography1973,geliEffectTopographyEarthquake1988,umedaHighAccelerationsProduced1987,gaffetSiteEffectStudy2000}. (2) The amplification at mountain tops is systematically larger for incident S compared with P waves, and such difference diminishes when the slope decreases or the incidence angle increases \citep{bardDiffractedWavesDisplacement1982}. (3) Body waves and surface waves are strongly scattered, thus reducing ground motion amplitudes while prolonging the shaking durations \citep{sanchez-sesmaDiffractionSVRayleigh1991,leeEffectsRealisticSurface2009}. (4) Topography tends to disrupts the coherency of high-frequency ground motion and thereby distorts the S-wave radiation pattern \citep{imperatoriRoleTopographyLateral2015}. 2D simulations are found to underestimate such amplification, due to lack of considering 3D effects \citep{geliEffectTopographyEarthquake1988,bouchonSeismicResponseHill1996}. While some of the following studies used 3D earth models to simulate ground motions above 1 Hz, e.g., \citet{rodgersBroadband04Hz2018} for the region at the North Korean Nuclear test site, and \citet{imperatoriRoleTopographyLateral2015} for the Swiss alpine region, they did not attempt to validate their results against recorded ground motion. Geometrical characteristics, such as smoothed curvature and relative elevation, have been explored to approximate topographic effects \citeg{maufroyFrequencyScaledCurvature2015,raiEmpiricalTerrainBasedTopographic2017}, but they require critical parameter constraints based on local velocity and target frequency, and thus difficult to be generalized for broadband studies.

In this study, we simulate ground motions for frequencies up to 5 Hz in the widely-tested SCEC CVM-S4.26M01 including high-resolution topography and compare to strong-motion data for the 2014 $M_w$ 5.1 La Habra, CA, earthquake, in order to constrain the relative contribution of topography, SSHs, and $Q(f)$. Our paper is organized as follows. We first describe the velocity model, simulation parameters, processing of the synthetic and recorded ground motions, and source description. Then the relative effects of model features such as topography, shallow near-surface velocities, small-scale heterogeneities, and $Q(f)$ will be quantified through goodness-of-fit (GOF) measures between synthetics and data, as guidelines for future simulations. Finally, we discuss future research directions based on our results.



%%%%%%%%%%%%%%%%%%%%%%%%%%%%%%%
\section{Computational Aspect}\label{approach}

In this section we describe the simulation method and model setup, and summarize the features included in our model.
\subsection{Numerical method for simulating ground motions}
We use the staggered-grid finite-difference (FD) code AWP-ODC \citep[Anelastic Wave Propagation, Olsen-Day-Cui, from the authors of the code, hereafter denoted by AWP;][]{cuiScalableEarthquakeSimulation2010}, which is $4^{\text{th}}$-order accurate in space and $2^{\text{nd}}$-order accurate in time to generate the ground motions for the La Habra event. AWP has been adapted to GPU accelerators for kinematic sources \citep{cui2013physics}, and provides support for frequency-dependent viscoelastic attenuation \citep{withersMemoryEfficientSimulation2015} and topography \citep{oreillyHighorderFiniteDifference2021}, which enable the simulations presented here. The original version of AWP employs a spatially uniform grid over the entire domain, where the minimum velocity determines the grid spacing, resulting in significant over-discretization in large-velocity areas. \citet{nieFourthOrderStaggered2017} developed a method that supports discontinuous mesh (DM) for the 3D staggered-grid finite difference scheme in AWP. The wavefields are exchanged within an overlap zone between two media partitions with a factor-of-three change in grid spacing, which significantly reduces the number of grid points needed in high-velocity regions and thereby improves the efficiency.

The accuracy of AWP has been thoroughly verified. For example, large-scale earthquake simulations in realistic 3D earth models with strong heterogeneities and complex faulting \citep{bielakShakeOutEarthquakeScenario2010,bielak2016verification}, revealed good agreement between AWP, another staggered-grid FD code and a finite-element code. The DM method implemented in the scalable GPU version of AWP was verified against uniform mesh solutions for the $M_w$ 5.1 La Habra earthquake \citep{rotenHighfrequencyNonlinearEarthquake2018}. \citet{oreillyHighorderFiniteDifference2021} implemented support for surface topography in AWP using a curvilinear grid, with its accuracy verified against SPECFEM3D.

\subsection{Computational Domain}
We used a model domain of lateral dimensions 148 km by 140 km dimensions, rotated 39.9$\circ$ clockwise with a depth extent of 60 km. The mesh was extracted from the SCEC CVM-S4.26-M01, and update of the original CVM-S4 model \citep{magistraleSCECSouthernCalifornia2000,kohlerMantleHeterogeneitiesSCEC2003} with iterative 3D full tomography inversions in Southern California \citep{leeRapidFullwaveCentroid2011}. The SCEC Uniform Community Velocity Model software framework \citep[V19.4][]{smallSCECUnifiedCommunity2017} was used for the extraction of seismic P-wave ($V_P$) and S-wave ($V_S$) velocities and the material density. The choice of CVM-S4.26-M01 (hereafter abbreviated with CVM-S) for this study was based on the results by \citet{tabordaEvaluationSouthernCalifornia2016} who concluded from a comprehensive validation of four velocity models with 30 earthquakes in the greater Los Angeles region that this model consistently yielded the best fit.

\subsection{Small-scale Heterogeneities}
Small-scale (on the order of tens to hundreds of meters) crustal heterogeneities are known to exist in nature but are insufficiently resolved in state-of-the-art velocity models. Instead, small-scale heterogeneities are commonly included in numerical simulations via statistical models of fluctuations in velocities and density \citeg{imperatoriBroadbandNearfieldGround2013,savranGroundMotionSimulation2019}. We superimpose a statistical model of the property perturbations onto CVM-S, defined via a Von Karman shape function \citep{frankelFiniteDifferenceSimulations1986}:

\begin{equation}\label{eq:highf-1}
  \Phi_{v, a}(r)=\sigma^{2} \frac{2^{1-v}}{\Gamma(v)}\left(\frac{r}{a}\right)^{v} K_{v}\left(\frac{r}{a}\right)
\end{equation}

which has Fourier transform:

\begin{equation}\label{eq:highf-2}
  P(k)=\frac{\sigma^{2}(2 \sqrt{\pi} a)^{E} \Gamma(v+E / 2)^{v+E / 2}}{\Gamma(v)\left(1+k^{2} a^{2}\right)}
\end{equation}
\noindent in which $k$ is the wave number and $E$ is the Euclidean dimension, $|Gamma$ denotes the gamma function, $K$ stands for the modified Bessel function of the second kind with order $\nu$. The parameters of the Von Karman autocorrelation function include correlation length $a$, standard deviation $\sigma$ and Hurst number $\nu$. This approach generates a random field with zero mean, and the desired standard deviation is guaranteed by scaling the random variable at each computational node. We used a fixed Hurst number of 0.05 and introduced elliptical anisotropy with a ratio of horizontal-vertical correlation lengths of 5. We tested correlation lengths between 100-500 m, and standard deviation of 5\% and 10\%, based on previous studies in Southern California \citeg{nakataStochasticCharacterizationMesoscale2015,savranModelSmallscaleCrustal2016}. In our model, the random perturbations extend to a depth of 7.5 km, and then linearly tapered down to standard deviation of 0 at 10 km depth \citep{liEvaluationOneDimensionalMultiDirectional2018}. Figure 2 shows an example realization of small-scale heterogeneities, compared to the original CVM-S in terms of $V_S$ at the surface. We did not endeaver to generate mutiple random field realizations of SSHs with the same parameters due to the computational limits and the report that a single realization is alike an ensemble of realizations when the spatial extent is large with a massive number of points \citep{hartzellEffects3DRandom2010}.

\subsection{Topography}
Despite the relatively flat topographic relief in the greater Los Angeles region, including near the epicenter of the La Habra event, the San Gabriel and Santa Ana Mts bound the area to the North and East, respectively (see Figure 1). To quantify the effects of topography on ground motions from the La Habra event, we use the curvilinear grid approach by \citet{oreillyHighorderFiniteDifference2021}. In this version of AWP, surface topography is incorporated by stretching the computational grids in the vertical direction, while keeping the horizontal grid spacing unchanged, so that the surface grid locations conform to the shape of the topography. We include surface topography into our model domain via the $\frac{1}{3}$ arc-second resolution Digital Elevation Model in southern California from the U.S. Geological Survey (USGS).

\subsection{Frequency-Dependent Attenuation}
Anelastic attenuation is needed for accurate simulation of seismic wave propagation through earth models at distances further than the dominant wavelength to account for the loss of intrinsic energy. Frequency-independent attenuation, resulting in identical seismic energy loss per cycle across a frequency bandwidth, has been used successfully in numerical simulation studies \citeg{bielakShakeOutEarthquakeScenario2010,savranGroundMotionSimulation2019}. However, as frequencies increase above about 1Hz, data often supports frequency dependence of $Q$ \citeg{raoof1999attenuation,eberhart2014imaging,wangUsingDirectCoda2017}. To address this observation, \citet{withersMemoryEfficientSimulation2015} developed an efficient coarse-grained memory variable approach to model attenuation quality $Q$ using a power law formulation:

\begin{equation}\label{eq:highf-3}
  Q(f)=Q_{0} *\left(\frac{f}{f_{0}}\right)^{\gamma}, f \geq f_{0}
\end{equation}
where $Q_0$ is frequency independent attenuation quality for $f<f_{0}$, and found that an exponent of 0.8 above 1 Hz produced a match with both ground motion data and Ground Motion Prediction Equations (GMPEs) up to 4 Hz, as compared to a constant $Q$ model.  \citet{savranGroundMotionSimulation2019} obtained optimal results for a $\gamma$ of 0.6 in their 0-2.5 Hz simulations for the 2008 Chino Hills, CA, earthquake.

A popular parameterization of $Q_0$ is proportional to local seismic velocity, with separate values $Q_P$ and $Q_S$ for $V_P$ and $V_S$ quality factors, respectively, producing the expected stronger attenuation for lower velocity material \citep{haukssonAttenuationModelsThree2006}. \citet{taborda2014ground}  revised the formula expressed by \citet{brocher2008compressional} and applied a $6^{\text{th}}$-order polynomial function for $Q_S$ and $Q_P=\frac{3}{4}\left(V_P/V_S\right)^2Q_S$. We test a variety of these parameterizations of $Q_0$ for the La Habra event.

\subsection{Near-surface Geotechnical Layer}
The CVM-S model includes geotechnical data which integrates geology and geophysics data from surficial and deep boreholes, oil wells, gravity observations, seismic refraction surveys and empirical rules calibrated based on ages and depth estimates for geological horizons in southern California \citeg{magistraleGeologybased3DVelocity1996,magistraleSCECSouthernCalifornia2000}. While validation studies, such as \citet{tabordaEvaluationSouthernCalifornia2016}, have shown that basin structures of CVM-S are reasonably accurate, unrealistically large surface rock site velocities (see Figure 2) motivated the method by \citet{elyVs30derivedNearsurfaceSeismic2010} to reduce the $V_S$ in the top 350 m based on available $V_{S30}$ values. Most recently, \citet{huCalibrationNearsurfaceSeismic2021} proposed an improved velocity tapering method that the top 1000 m shallow velocities are tapered to fit the Fourier spectra of the records below 1 Hz.

\subsection{Ground motion simulations}
Table 1 lists the simulation parameters used in our simulations. All simulations have the same duration of 120 s and resolve wave propagation up to $F_{max}=5$ Hz by at least 5 points per minimum S-wavelength. We use two different codes: 1) AWP-DM with a flat free surface, and 2) AWP-topo with a regular, curvilinear mesh. AWP-DM was used to quantify the effects of the near-surface material with a minimum $V_S$ of 200 m/s using 3 different mesh discretations: 1) dx=8 m from the surface to 1,472 m, 2) dx=24 m between 1,472 m and 10,336 m, and 3) dx=72 m at deeper levels. We used a kinematic source generated following \citet{gravesKinematicGroundMotion2016}, which creates finite-fault rupture scenarios with stochastic characteristics optimized for California events. The focal mechanism was taken from the U.S. Geological Survey \citep[strike=233°, dip=77°, rake=49°; ][]{usgsEarthquakeEventsFocal2014} with a moment magnitude 5.1, fault area of 2.5 km x 2.5 km.

\subsection{Data Processing}
259 strong-motion seismic stations were used to validate the simulations. The strong motion recordings (velocity time series) are obtained from SCEC (F. Silva, Personal Communication, 07/2020), with hypocentral distance up to 90 km and signal-to-noise ratio above 3 dB. The processing procedure included the following steps: (1) low-pass filtering of the time series below 10 Hz using a zero-phase filter; (2) interpolating the time series linearly to a uniform time step; (3) tapering of at the last 2 seconds using the positive half of a Hanning window; (4) zero padding the last 5 seconds; (5) filtering the seismograms to the desired frequency, and (6) converting velocities to accelerations by a time derivative. Except for the initial 10 Hz low-pass filter, all filters used a low-cut frequency of 0.15 Hz to avoid noise interference. $4^{\text{th}}$-order Butterworth filters were used in all cases. Finally, our horizontal synthetic seismograms were rotated 39.9° clockwise.

\subsection{Goodness of fit criteria}
We used a modified subset of the goodness-of-fit metrics proposed by the methods of \citet{andersonQuantitativeMeasureGoodnessOfFit2004} and \citet{olsenGoodnessoffitCriteriaBroadband2010}for comparison of broadband seismic traces (0 to 10+ Hz). The method includes 10 different metrics, from which we selected 7, namely peak velocity (PGV), peak acceleration (PGA), energy duration (DUR), cumulative energy (ENER), response spectral acceleration averaged between 0.1 and 10 s (RS), and smoothed Fourier amplitude spectrum (FAS), as well as one additional metric, Arias intensity (AI), to compute our GOF scores. We computed the SA at frequencies linearly spaced from 0.2 to 5 Hz. Cumulative energy is calculated as the integral of the square of velocity instead: $E N E R=\int v(t)^{2} d t$. We define the DUR as the time between the arrival of 5 and 95 percent of the total energy and ENER is calculated in this window. The Arias intensity describes the cumulative energy per weight \citep{arias1970measure}, and is defined as $AI=\frac{\pi}{2 g} \int a(t)^{2} d t$, where $a(t)$ is the acceleration time series, $g$ is the gravitational acceleration and the integral is again performed in DUR. The GOF score is defined as
\begin{equation}\label{qe:highf-4}
  G_{\text {metrics }}=10 \operatorname{erfc}\left(\frac{2|x-y|}{x+y}\right)
\end{equation}
\noindent where $x$ and $y$ are two positive scalars from the metrics we selected. $G_{\text {metrics }}$ is computed for each metric and then averaged assuming equal weights combining all selected metrics except for PGV and PGA (measuring peak amplitudes), ENER and AI (measuring the time-progressive intensities), which show strong correlations within each pair and thus their weights are halfed. $G_{\text {metrics }}=\frac{1}{N} \sum G_{\text {metric }}$ and three components $G_{\text {station }}=\frac{1}{3} \sum_{i=1}^3 G_{\text {metrics}, i}$, where $i$ represents three components. The GOF score for the entire simulation is calculated at the average of Gstationacross all 259 stations. In general, GOF values between two signals above 4.5 and 6.5 are considered fair and very good fit, respectively.

\section{Results}
\subsection{Sourcel models}
Due to the stochastic characteristics of the kinematic source generator by \citet{gravesKinematicGroundMotion2016}, a series of source realizations with different random seeds are evaluated based on comparisons between spectral accelerations with records at stations within 31 km to the epicenter (R. Graves, Personal Communication, 03/04/2020; see Figure 3). We selected three best-performed source descriptions, with varying hypocentral depths at 5, 5.5 and 6 km, respectively (see Figure 3). The rupture durations are less than 2 s and sampled at an interval of 0.001 s, identical to the time step used in our simulations. The three sources tend to generate overall similar patterns of PGV for both low (< 1 Hz) and high (> 2.5 Hz) frequencies (Figure 4), while Source 2 and 3 tend to generate larger amplification in the Chino Basin and toward the south, respectively. Based on this result, we limit the subseqeent analysis to using Source 1 only.

\subsection{Minimum $\mathrm{V_S}$}
Southern California features numerous low-velocity basins where the minimum (surface) $V_S$ in CVM-S can be much lower than the minimum value of 500 m/s that we used in our domain 1 (see Table 1). Previous studies have pointed out that soft soils, characterized by lower $V_S$ can cause significant amplification of the ground motions \citeg{anderson1984model}. To examine the effects of the low-velocity near-surface material, we repeated our reference simulation with a minimum $V_S$ of 200 m/s, see Figure 5. As expected, the largest effects are observed above the basin areas that are highly correlated with the low near-surface velocities. The effect on PGV is almost an order of magnitude larger than that for DUR. At frequencies below 1 Hz, the PGV amplification is generally concentrated in basins with low near-surface Vs. However, as the frequency increases, the amplification in PGV extends to the surrounding low areas, including mountain valleys. For 2.5-5 Hz seismic waves, deamplification starts to show up in mountain regions, likely due to entrapment and attenuation of seismic seismic energy in the basin areas. While the effects of the near-surface velocities are somewhat similar for PGV and AI in pattern and amplitude, DUR shows much smaller effects.

Figure 6 shows the effects of the near-surface low $V_S$ quantitatively. While the low velocities cause up to 50\% local amplification in PGV, the average effect is mostly less than 10\%. Both the mean and standard deviation of the difference between the two models increase with the frequency, but remain at a relatively low level up to 5 Hz. Thus, our results suggest that despite locally large effects in particular for the higher frequencies, omitting the effects of material with Vs< 500 m/s does not change the overall ground motions significantly. Based on this result, we adopt a minimum $V_S$ of 500 m/s in the remaining analysis to alleviate computational cost.

\subsection{Reference Model Response}
Figure 5 shows a comparison between PG$V_S$ extracted from synthetics and data for a simulation with source 1, CVM-S including topography, $Q(f)=0.1f^{0.6}$, $Q_P=2Q_S$, shallow velocity taper down to 1000 m, and no small-scale heterogeneities added, which is our reference case for low frequencies (0.15-1 Hz) and high frequencies (2.5-5 Hz). The minimum $V_S$ is truncated at 500 m/s (at which Vp is calculated from the Vp/$V_S$ ratio in CVM-S, and density is unchanged), often necessitated by limitations of computational resources. While the overall PGV distributions are similar, the simulation tends to predict larger values to the south and west, primarily for the high frequencies. The simulated PG$V_S$ overpredict the basin amplification in the Ventura, Chino and San Bernardino basins (see Figure 2a for locations) below 1 Hz, particularly for distances 10-30 km, while the synthetic high-frequency PG$V_S$ consistently underpredict those from the data (see Figure 5c) by more than 50 \% at distances more than about 40 km away from the source. The comparison suggests a slightly lower $Q$ below 1 Hz, and a significantly higher $Q$ at epicentral distances further than 40 km above 2.5 Hz.

\subsection{Topography}
In this section we investigate the effects of topography, which are often ignored in numerical simulations \citeg{graves2004observed,olsenStrongShakingAngeles2006,savranGroundMotionSimulation2019}. Our analysis of topographic effects is based on the reference model with topography removed.
Figure 8 shows the percent difference between models with and without topography for PGV and DUR and bandwidths of 0.15-1 and 2.5-5 Hz. It is clear that topography complicates the wavefield pattern significantly, even at frequencies below 1 hz. Consistent with previous studies \citeg{hartzell1994initial, leeEffectsRealisticSurface2009} we see a general de-amplification of PGV below 1 Hz, and that mountain peaks and ridges tend to amplify PGV by 30-60\%. Also, PGV is reduced by about 30\% in the Chino Basin and Northwest of San Gabriel Mountains but remains almost unchanged in other regions.[Why is that?] The strongest area-wide effect of topography below 1 Hz is the increase of DUR by about 40\%.
As the frequency increases above 2.5 Hz, localized areas of PGV amplification appear in the western Transverse Ranges, while areas north of the Santa Ana and San Gabriel Mountains show strong reduction in PGV. [Why is that?] Also, \citet{leeEffectsRealisticSurface2009} noticed that the effects from topography interfere with those from path and directivity. Starting from the epicenter towards northwest, they noticed a pattern of “amplification-deamplification-amplification” is observed, which is consistent with the numerical study by \citet{liuScatteringSeismicWaves2020}, showing that crests and troughs with strongly varying amplification appear along the short axis of a hill as frequency increases. We interpret these results as shielding and focusing effects on the front side and back side of the hills, respectively, becoming more significant at higher frequencies, in agreement with \citet{liuScatteringSeismicWaves2020}.
Our results show a strong negative correlation between the effects on PGV and DUR, suggesting that topography redistributes seismic energy from the large-amplitude first arrivals to the adjacent coda waves, particularly at the higher frequencies. This trend causes the ground motion variability to increase in the presence of complex topography, generating stronger regional concentration of amplification and deamplification at higher frequencies. The fractal structure of topography, causing interaction of seismic waves at different wavelengths, greatly complicates interpretation of its effects on the ground motion at frequencies, as noted by \citet{booreNoteEffectSimple1972} and \citet{panzeraEvidenceTopographicEffects2011}.

\subsection{Frequency-Dependent Attenuation}
Figure 9 shows the horizontal-component FAS generated from three different attenuation models, namely $Q_S=0.1V_S$ (constant $Q$), $Q_S=0.05V_Sf^{0.6}$, $Q_S=0.1V_Sf^{0.6}$. The $Q(f)$ models and the constant $Q$ model diverges above 1 Hz as expected, with more high-frequency energy sustained in the former, resulting in a better match to data. Among the three attenuation models, $Q_S=0.1V_Sf^{0.6}$ best predicts the FAS of the data, while $Q_S=0.05V_Sf^{0.6}$ underpredicts data from 0.2 to 5 Hz. The finding that a velocity coefficient of 0.1 is optimal in Southern California with the CVM-S model is consistent with \citet{savranGroundMotionSimulation2019}.

Despite the significant improvement in FAS in high frequencies above about 2.5 Hz for the frequency-dependent $Q$ models, PGV at distance farther than 40 km from the source is still underpredicted (Figure 10). The most overpredicted regions are generally the base of the hills, while the most underpredicted areas are the valleys behind the hills, suggesting that the discrepancies are due to omission of topography effects. [So, you are NOT including topography here? If you are including topography, you can’t blame the misfits on topographic effects. Are these misfits related to (lack of) $V_S$ refinement? If so, you should rerun and show results with refinement.]

\subsection{Small-scale heterogeneities}
Figure 11 shows the effects of adding a von Karman distribution of small-scale heterogeneities with $\sigma$ = 5\% and horizontal correlation length of 500 m to CVM-S to the reference model on PGV and DUR the frequency bands 0.15-2.5 Hz and 2.5-5 Hz. Our results indicate that the effects of small-scale heterogeneities on ground motions is second-order to other model features, in agreement with the results by \citet{savranGroundMotionSimulation2019} for the same general region. The effects of the small-scale heterogeneities increase with frequency, as illustrated in Figure 12 with respect to the reference model for PGV and DUR. Strong localized amplification and de-amplification for the frequency band 2.5-5 hz generates strong ground motion variability. The resulting PG$V_S$ and DURs at 0.15-2.5 Hz are very close to those for 0.15-1 hz. The ground motion effects generated by a series of different von Karman distributions of small-scale heterogeneities, with results similar to those in Figure 11 and 12,  are shown in the supplementary material.
Since only a single distribution of small-scale heterogeneities was applied, there is no clear spatial pattern between SSH effects and geological structure, and scattering is observed even within a few kilometers from the source. The small-scale heterogeneities generate areas of both amplification and deamplification, with stronger effects from distributions with larger standard deviations $\sigma$ and longer correlation length a, as expected. However, in general, small-scale heterogeneities tend to decrease ground shaking amplitudes, and prolong the shaking durations (e.g., near the San Gabriel Mountains and Santa Ana Mountains), indicating that the resulting scattering redistributes the seismic energy from main arrivals to the later coda waves.
\citet{przybillaEstimationCrustalScattering2009} used elastic radiative transfer theory to show  that the direction dependence of scattering can be identified by $ak$, where $a$ is the correlation length and $k$ is the wave number. For $ak \approx 1$, waves interact with heterogeneous medium most intensively because the wavelength and correlation length are nn the same order. When $ak\gg 1$ waves are predominantly scattered in the forward direction, which generates geometric focusing in the early arrivals and leads to larger peak amplitudes, and vice versa for $ak \ll 1$. As the frequency increases, $ak$ tends to increase as well, causing scattering effects that are different from low-frequency behavior. For example, the Santa Ana Mountains result in akmoderately larger than 1 [why?] and experience strong scattering effects that amplify the peak ground motions, decreasing toward the northeast ak as decreases [why?]. Within the basins with low near-surface velocities and resulting higher ak, this trend becomes less clear.

%%%%%%%%%%%%%%%%%%%%%%%%%%%%%%
\section{Discussion}
With the effects of various model features on ground shaking discussed above, we have reached a point to further explore whether it is possible and how we are able to combine theses known effects into a sophisticated model to better represent the realistic crust structure, and therefore the characteristics of ground motions extending to a significantly higher frequency range than previous  physics-based simulations. It is noteworthy that our simulation process has been verified with a set of other methods during  the SCEC High-F exercise as well as previous studies using AWP, which implies that our model construction and mathematical approach are reliable.

\subsection{Anelastic attenuation and $\kappa$}
The shape and amplitude of the Fourier amplitude spectrum (FAS) of strong ground motions reveal fundamental information about the physical processes of source rupture and wave propagation through the earth crust. At high frequencies above the corner frequency up to the noise level of Nyquist frequency, the FAS is observed to decay exponentially \citep{anderson1984model}, which can be described by the spectral decay parameter $\kappa$. Anderson and Hough found consistent $\kappa$ values in the range of 0.04-0.07 for Southern California, and only weak distance dependency that $\kappa$ increases with distance. The authors endorsed the hypothesis by \citet{hanksFmax1982} that the behavior of $\kappa$ is primarily caused by the subsurface geological structure close to the stations.

With the processed synthetics and records, we calculated $\kappa$ from a least-square fit to the spectra, taking the average of the two horizontal components, between 2-5 Hz in frequency-log (FAS) space. The spectra is then stacked over either the whole set of stations or the stations grouped into distance bins defined by the hypocenter distance to reduce the high-frequency oscillations in FAS at a single station. The frequency bandwidth was selected so that spectral decay starts to show up yet below the highest frequency our simulations can resolve.

Figure 14 shows the stacked FAS for a series of models with different attenuation models and/or topography. The models with frequency-independent attenuation decay more rapidly and thus yield larger $\kappa$ than data; while frequency-dependency produces a slower decay as frequency increases. As a result, the models with frequency-independent attenuation tend to underpredict the FAS at higher frequencies (above 1-1.5 Hz), which is improved for $Q_S=0.05V_Sf^{0.6}$, and overprediction is even observed when the coefficient further increases from 0.05 to 0.1. Topography generates small increases on the FAS between 2.5-3.2 Hz and then decreases between 3.2-4 Hz.

Our calculated $\kappa$ from data at the entire set of stations is 0.037, which is slightly smaller than that measured by \citet{anderson1984model}. The single $\kappa$, though representing an across-domain attenuation characteristics, is oversimplified and neglects site-specific information when our domain contains two distinctive geological structures, i.e. mountains and valleys, and thus rock types. It is natural to calculate $\kappa$ for each subset of stations grouped by their distance to the source to ensure the similarity among each group of stations. Figure *b plots the stacked FAS for the same models as in panel (a) and the corresponding $\kappa$ values as a function of $R_{hypo}$. The models with a power-law attenuation are in very good agreement with data. In addition, we observed the increase in $\kappa$ with distance, as reported by Anderson and Hough, up to 70 km rupture distance. Farther than 70 km, the fit with data is still good for attenuation models, though the $\kappa$ values start to decrease. This is not mentioned in previous studies, however, we note that beyond this distance, the number of stations in each bin is as small as only one or two, and thus the resulting $\kappa$ may be of low confidence due to too sparse data points. Topography, though complicated wave propagations and site response, has little influence on $\kappa$. Among various attenuation models, we found that $Q_S=0.05V_Sf^{0.6}$ produces the best fit of $\kappa$. We, however, note that the lowest $Q$ values in the model, representing the strongest attenuation, are also determined by the choice of the minimum $V_S$ in the model, therefore the model with larger minimum $V_S$ may tend to select a smaller coefficient. The good match obtained by frequency-dependent attenuation models indicates that frequency dependency itself is critical in improving the fit.

\section{Conclusions}
Based on accessibility of more accurate surface elevation information, mathematical descriptions of small-scale heterogeneities and frequency-dependent attenuation and community efforts in assessing shallow material structure, we explored various features and the complicated trade-offs between them. We have shown that physics-based deterministic numerical simulations are capable of capturing the most important characteristics of the ground motions incited by the $M_w$ 5.1 2014 La habra earthquake. Our results show that such high-frequency simulations can be  a valuable complement to the stochastic models that evaluate seismic hazard up to 25 Hz. Because of our great advance in pushing up the frequency range of deterministic simulations (from 1-2 Hz in most previous simulations to 5 Hz), the uncertainties in stochastic methods generating higher-frequency synthetics can be reduced significantly.
The fact that the GOFs at lower frequencies tend to be higher than at high frequencies is attributed to lack of enough high-resolution geological structure. The best geologically-constrained fine structure is surface topography, which increases peak ground motions at mountain tops and yields stronger effects at higher frequency, especially decreasing the ground shaking amplitude and prolonging the duration at low reliefs. The inclusion of more realistic low velocities is of relative minor importance, due to its sparsity in the model. However, stations underlain by very soft soils, with $V_S$ as low as 200 m/s, can still experience at most 80\% overprediction in PGV that is not captured when velocities are clamped at a higher value, e.g. 500 m/s. Small-scale heterogeneities (SSHs), generally reduce the peak ground motions and increase the durations, in a way similar to that of topography. More importantly, the models with SSHs always generate a better fit in duration compared to the model without SSHs. The frequency-dependent attenuation mainly amplifies the ground motions, and is critical in fitting the spectra decay according to our examination of the high-frequency decay ($\kappa$) as a function of distance.

Generally speaking, the measurements related to peak values, response spectra and duration are almost always best predicted; while the fits of the Arias intensity, energy and Fourier amplitude spectra are more difficult to improve. Furthermore, among the poor measures, the large standard deviations of Arias and energy across various models indicate the even greater difficulty in fitting, in contrary to Anderson’s (\citeyear{andersonQuantitativeMeasureGoodnessOfFit2004}) speculation; while the Fourier spectra shows a very low standard deviation, which implies some systematic misfit. By examining the velocity profiles and FAS at rock sites, we found the unrealistic near-surface velocities in the CVM-S likely explains this misfit. Furthermore, introducing near-surface low velocities, e.g. via a $V_{S30}$-based method, can greatly improve the underprediction. [[Brief description of the refinement model up to 5 Hz]]
With the effects and importance of the feature summarized, a subset of our well-performed models with these features included, achieve a fair improvement in terms of the proposed GOF criterion. In contrast, the simplest models without incorporating any additional features yield consistently lower GOF scores with the increase of frequency. Our results suggest that the major source of misfits is the poor constraints in the underlying velocity model at the high resolution required for our high-frequency simulations.  The composite models that generated the best fit suggests that incorporating topography, frequency-dependent attenuations (e.g. $Q_S=0.05V_Sf^{0.6}$ or $Q_S=0.1V_Sf^{0.6}$) and SSHs is a valid supplement to the velocity model in numerical simulations.
In the future, we expect advancements in community velocity models, with further calibrated features that are explored in this study, can help improve the fit to data. In addition, the computational cost is still a barrier for us to perform more detailed investigations, which will be partly solved once the AWP code with discontinuous mesh and topography is available. It will be possible to extend such studies to larger earthquakes, possibly with nonlinear effects, which remains a major concern for residents in Southern California.


\section*{Data and Resources}
The UCVM program used to extract velocity meshes can be obtained from SCEC on \url{https://github.com/SCECcode/UCVMC} (last accessed 12/2020). The simulations were performed on Summit at the Oak Ridge Leadership Computing Facility in Tennessee. Most of the data-processing work was done using Python and the Generic Mapping Tools package (\url{https://www.generic-mapping-tools.org}, last accessed 04/2021).


\section*{Acknowledgements}
\addcontentsline{toc}{section}{\protect\numberline{}Acknowledgements}

This research was supported through the U.S. Geological Survey External Program (award \#G19AS00021), as well as the Southern California Earthquake Center (SCEC; Contribution Number xx). SCEC is funded by the National Science Foundation (NSF) Cooperative Agreement EAR-1600087 and the U.S. Geological Survey (USGS) Cooperative Agreement G17AC00047. We thank Robert W. Graves for providing the source models and Fabio Silva for providing the station records of the 2014 La Habra earthquake.

\Cref{chap:highf}, in full, is a reformatted version of a paper currently being prepared: Hu, Z., Olsen, K.B. and Day, S.M. (2021), 0-5 Hz Deterministic 3D Ground Motion Simulations for the 2014 $M_w$ 5.1 LaHabra Earthquake. The dissertation author was the primary investigator and author of this paper.

\newpage
\section*{Tables and Figures}
\addcontentsline{toc}{section}{\protect\numberline{}Tables and Figures}%

%% For very long table
% \clearpage
% \begin{sidewaystable}[!ht]
% \caption{Coregionalization matrix $\mathbf{P}^\mathbf{3}$}
% \begin{adjustbox}{width=\textwidth,center}
% \begin{tabular}{|c|cccccccccccccccccccccccccccccccc|c|}
% \end{tabular}
% \label{tab:5-S3}
% \end{adjustbox}
% \end{sidewaystable}



% %%%%%%%%%%%%% figures 

\clearpage
\floatsetup[figure]{style=plain,subcapbesideposition=top,font=Large}
% \captionsetup[figure]{justification=justified,
%     labelfont={color={magenta},bf},textfont={color={green}},
%     labelsep=newline}
\begin{figure}[!ht]
  \sidesubfloat[]{\includegraphics[width=0.9\textwidth]{topo_q100f06_ely1000_hist.pdf}\label{fig:highf-7a}} \\[1.2\baselineskip]% %\hfil
  \sidesubfloat[]{\includegraphics[width=0.9\textwidth]{topo_q100f06_hist.pdf}\label{fig:highf-7b}} %% \\[\baselineskip]% $$\hfil
  \caption{ (a) $V_S$ profile sample locations in California. Triangles denote rock sites and circles denote soil sites, and (b) extracted $V_S$ profiles. The top panel zooms into the top 500 m. }
  \label{fig:highf-7}
\end{figure}

% \clearpage
% \floatsetup[figure]{style=plain,subcapbesideposition=top,font=Large,footfont=Large}
% \begin{figure}[!ht]
%     \sidesubfloat[]{\includegraphics[width=0.4\textwidth]{figures/figure_vs30_14a.png}\label{fig:vs30-14a}} \hfil
%     \sidesubfloat[]{\includegraphics[width=0.45\textwidth]{figures/figure_vs30_14b.pdf}\label{fig:vs30-14b}} % \\[\baselineskip]%
%     \caption{ (a) $V_S$ profile sample locations in California. Triangles denote rock sites and circles denote soil sites, and (b) extracted $V_S$ profiles. The top panel zooms into the top 500 m. }
%     \label{fig:vs30-14}
% \end{figure}
% \clearpage

%% supplement
\setcounter{table}{0}
\setcounter{figure}{0}
\numberwithin{figure}{chapter}
\numberwithin{table}{chapter}
\renewcommand{\thetable}{S\arabic{chapter}.\arabic{table}}
\renewcommand{\thefigure}{S\arabic{chapter}.\arabic{figure}}
\newpage
\section*{Supplementary Materials}
\addcontentsline{toc}{section}{\protect\numberline{}Supplementary Materials}

This supplement includes.




\renewcommand{\thetable}{\arabic{table}}
\renewcommand{\thefigure}{\arabic{figure}}

\numberwithin{figure}{chapter}
\numberwithin{table}{chapter}

%\endrefsection
