% !TEX encoding = UTF-8 Unicode

\linespread{1.7}
\chapter{Introduction}
\linespread{2.0}
\label{chap:intro}

\section{Motivation}
Earthquakes, caused by energy release from the Earth's interior during short time span, constitute one of the most catastrophic natural hazards to human society. While the majority of earthquakes are too small to be felt by humans, their recordings from growing sensor deployments increasingly contribute to our understanding of seismology. Major earthquakes with magnitude greater than 7 occur globally more than once per month. The timing of these events remains unpredictable, with limited progress expected in the near future. On the other hand, considerable progress has been made in estimation of the spatial distribution of ground motions for future damaging earthquakes, an important ingredient in seismic hazard analysis.

Ground motion prediction is central to seismic hazard analysis, because ground shaking from major earthquakes is oftentimes the most dominant source of direct damage, as well responsible for secondary effects, such as tsunamis, landslides, liquefaction and fires. A conventional approach to ground motion estimation is to perform statistical analysis from earthquake records, incorporating correlations between earthquake characteristics and measurements. This method is referred to as ground motion models (GMMs) or ground motion prediction equations \citep[GMPEs; e.g., ][]{abrahamsonSummaryASK14Ground2014,booreNGAWest2EquationsPredicting2014,campbellNGAWest2GroundMotion2014,chiouUpdateChiouYoungs2014, idrissNGAWest2EmpiricalModel2014}, typically providing median values with uncertainty estimates. However, due to lack of records for large earthquakes and at near-fault sites (e.g., less than about 10 km), estimates of strong ground motions using GMPEs, especially within close distances to the earthquake source, suffer from large uncertainty. Physics-based numerical methods represent viable alternatives to produce ground motion estimates for large earthquakes and near-fault distances required for accurate seismic hazard analysis, public earthquake emergency preparedness, and the design of seismic-resistant structures in civil engineering.

As analytical solutions for wave propagation in 3D velocity structures are often intractable, physics-based deterministic simulations solving the elastodynamic equations numerically are required to obtain ground motion estimates for realistic scenarios. A variety of deterministic numerical methods have been developed used for gorund motion prediction, including finite-difference, finite-element, boundary element, spectral element and the pseudo-spectral methods \citeg{sanchez1982boundary,frankelThreedimensionalSimulationSeismic1992,olsenThreeDimensionalSimulationMagnitude1995,gravesSimulatingSeismicWave1996,baoLargescaleSimulationElastic1998,seriani3DLargescaleWave1998,kaserArbitraryHighorderDiscontinuous2006, chaljub2006spectral}. These numerical methods differ from each other with individual advantages or disadvantages in terms of accuracy, efficiency and ranges of applicability.  Based on extensive verification and validation, though under various simplifications and assumptions due to insufficiently resolved parameters and computational limits, these methods have been developed and applied successfully in a series of meaningful research exercises \citep[e.g., ShakeOut, PetaShake, M9 Cascadia; for more details readers are referred to ][]{cuiPetascaleEarthquakeSimulations2009,cuiTeraShakeComputationalPlatform2009,olsen2009shakeout,marafiImpactsSimulatedM92019}. Among these methods, the finite differences is the most widely used method in large scale three-dimensional (3D) simulations considering its simplicity in mesh generation and GPU parallelization, efficiency and scalability. Additionally, finite-difference methods are able to support nonlinear soil behavior, frequency-dependent anelastic attenuation, discontinuous mesh discretization and irregular surface topography \citeg{rotenQuantificationFaultZonePlasticity2017, nieFourthOrderStaggered2017,zhangThreedimensionalElasticWave2012,oreillyHighorderFiniteDifference2021}.

Armed with recently accelerated availability of high-performance computing resources, researchers have achieved considerable progress in higher-frequency ground motion simulations in the last two decades \citep{gravesBroadbandSimulationsSouthern2008,olsen2009shakeout,bielakShakeOutEarthquakeScenario2010,roten3DSimulationsEarthquakes2012, savranGroundMotionSimulation2019,withersGroundMotionIntraevent2019}. Despite of this progress, it is critical to keep refining the temporal and spatial characteristics of broadband seismic wave propagation simulations, in order to make further advances in mitigating life and property in future damaging earthquakes. In general, deterministic earthquake simulations involve three main aspects, namely source, path, and site effects, covering a broad range of fields including rupture nucleation, rock failure, plastic deformation, attenuation and scattering, 3D path effects, low-velocity amplification, etc. This thesis explores the relative contributions to ground motion estimation from a series of physics-based model features.


\section{Near-source Plasticity}

A number of densely populated regions are located close to major faults where extreme earthquakes are likely to occur, e.g., Los Angeles, San Francisco, Tokyo, Istanbul and others. Seismic hazard analysis and building code design for these regions are particularly affected by the poorly constrained database of near-source ground motions which leads to large uncertainty of the high ground motion levels at low exceeding probability \citeg{hanksObservedGroundMotions2005, bommerWhyModernProbabilistic2006}.  Oftentimes, large-scale simulations consider linear deformation for modeling simplicity \citep{olsen20083d,molnar2014earthquake}, while studies have shown that experience nonlinear effects in large earthquakes, which typically decreases the peak ground motions close to the fault \citep{andrewsPhysicalLimitsGround2007,ma2008physical,duanSensitivityStudyPhysical2010,templetonDynamicRuptureBranched2010,dunhamEarthquakeRupturesStrongly2011}. Physics-based numerical simulations have the potential to the sparse observations in the near-fault regions. While nonlinear soil effects are traditionally found for frequencies above 1 Hz based on 1D analysis \citet{fieldNonlinearSiteResponse1998},  \citet{gravesBroadbandSimulationsSouthern2008} found from 3D deterministic simulations that visco-plastic rheology can reduce the peak amplitudes of long-period (<1 s) ground motions by up to 70\% in the Los Angeles Basin compared to the response obtained with visco-elastic models. 

% More to be added from the EKS draft
%The numerical simulations provide an viable approach to establish physical limits in predicting ground motions in such regions. softening and damping


\section{Model Characteristics in High-Frequency Simulations}

State-of-the-art velocity models used in seismic wave propagation simulations are typically based on geological, geotechnical and geophysical information, and the accuracy of these velocity models are critical in minimizing systematic uncertainties in the resulting ground motion predictions. In the past decades, most large-scale deterministic earthquake simulations have focused on low-frequency ($f_{max}$ $\leqslant$ 1 Hz) seismic waves, which allows relatively coarse quantification of model properties. Generally, these simulations excluded surface topography, small-scale spatial heterogeneities and frequency-dependent attenuation, along with the minimum velocity clamped at often unrealistically high values to alleviate computational costs. However, these model features are expected to play an increasingly important role when entering the high-frequency bandwidth, along with trade offs complicating their relative contributions to the resulting ground motion estimates. For example, scattering increasing the envelope duration and often lowering the peak amplitudes is generated by both topography and velocity heterogeneities \citet{laiShallowBasinStructure2020}.
In the following sections, we briefly discuss the physics-based deterministic simulation features features and how they can be accommodated in high-frequency deterministic simulations.


\subsection{Near Surface Low Velocity}
The presence of near-surface low-velocity layers can have a dramatic effect on the simulated ground motion \citeg{imperatoriRoleTopographyLateral2015}.
%\citet{shawUnifiedStructuralRepresentation2015} investigated the effects of different velocity models and found significant sensitivity of ground motions to the velocity model. 
Previous studies have shown that low near-surface material properties, in particular the shear-wave velocity $V_s$, can lead to significant amplification, especially at higher frequencies, e.g. 0.5-10 Hz \citep{booreSiteAmplificationsGeneric1997,poggiDerivationReferenceShearWave2011}. In addition, low-velocity layers can increase coda wave amplitude and duration by trapping energy close to the free surface \citet{imperatoriBroadbandNearfieldGround2013}. Unfortunately, many large-scale deterministic simulations, in particular those using uniform grid approaches, have had to adopt larger than realistic surface low velocities, because of the limitation of computing resources.

Due to the oftentimes significant amplification effects, characterization of shallow material properties is an important task for ground motion estimation. Numerous well-established techniques have been applied for local characterization of the shallow velocities, including seismic borehole logging, surface-wave dispersion survey, cone penetration test, gravity observations and oil-well samples. The regional-scale characterization, however, is technically infeasible and financially prohibitive. For this reason, ground motion models typically rely on information compiled into 3D crustal velocity models, such as the Community Velocity Models (CVMs) maintained by the Southern California Earthquake Center (SCEC). The CVM velocities below the top 1-2 kilometers are generally relatively well-constrained from tomographic inversions \citeg{leeTestingWaveformPredictions2014,shawUnifiedStructuralRepresentation2015}, while the near-surface velocity information consists of a combination of geotechnical information and the S-wave velocity in the very uppermost 30 m ($V_{s30}$) when available, along with empirical functions, derived from features such as topography (e.g., Wald ...). However, the velocities between the top 30 m and the tomographic constraints are relatively poorly determined. \citet{elyVs30derivedNearsurfaceSeismic2010} proposed a method to include the $V_{s30}$ information into the CVMs via tapering to the values at a fixed threshold depth. We find that this threshold depth is poorly determined, and we propose an update to this method for the greater Los Angeles basin based on our modeling of the 2014 M5.1 La Habra, CA, earthquake (see \cref{chap:vs30}). 

%Different SCEC CVMs, e.g., CVM-S4.26M01 and CVM-H15.1, all using tomographic and source inversions as base models with basin information incorporated, show similarly poor resolution in mountainous areas where geological measurements are sparse due to prohibitive cost. We expect our proposed shallow velocity refinement to achieve similar improvements in ground motion predictions at these rock stations.


\subsection{Topography}
Probably the most prominent effect of surface topography on seismic motions is to amplify ground motions at mountain ridges and peaks [as evidenced from both observational and numerical studies?] \citep{celebiTopographicalGeologicalAmplifications1987,kawaseTopographyEffectCritical1990,massaExperimentalApproachEstimating2010,burjanekEmpiricalEvidenceLocal2014}. However, a key challenge in exploring topographic effects is to isolate the contribution of local relief from other factors, such as stratigraphy, the presence of fault damage zone or low-velocity surface layers. For example, the assumption that the near-surface geology is consistent across areas with simple and highly irregular topography  \citep{celebiTopographicalGeologicalAmplifications1987,geliEffectTopographyEarthquake1988,chavez-garciaComplexSiteEffects2000} is difficult to make. Numerical simulations provide a convenient means to circumvent such assumptions and isolating topographic amplification \citep{booreNoteEffectSimple1972,sanchez-sesmaDiffractionSVRayleigh1991,lovati2011estimation,hartzellGroundMotionPresence2017}.

 A wide range geomorphic parameters have been used to describe topography geometry, including slope, curvature, relative elevation, and surface roughness \citep{ashfordAnalysisTopographicAmplification1997,nguyenEvaluationSeismicGround2007,bouckovalasNumericalEvaluationSlope2005}. These studies found that smoothed curvature and relative elevation, which are linearly correlated with each other, are key parameters controlling topographic amplification \citep{maufroyFrequencyScaledCurvature2015,raiEmpiricalTerrainBasedTopographic2017}. Because the characteristic length is critical and dependent on frequency, the topographic effects behave differently in different frequency bands. The frequencies affected by topographic amplification of ground motions is correlated with the characteristic dimensions of the relief. The conventional assumption is that topography has negligible effects below 1 Hz \citep{booreNoteEffectSimple1972, pischiuttaTopographicEffectsHill2010}, with increasing effects for higher frequencies. Check also Ma . Here, we will use the support for topography in our numerical modeling code to further isolate topographic amplification of ground motion for the greater Los Angeles basin due to the 2014 M5.1 La Habra, CA, earthquake.


\subsection{Anelastic attenuation}

Anelastic (intrinsic) attenuation of seismic waves propagating thorough Earth's crust can be described by the quality factor $Q$ as a fractal energy loss per cycle: $Q^{-1}(\textbf{x}, f) = \sigma E / 2\pi E$, where $\textbf{x}$ is the position, $f$ is the circular frequency and $E$ is the total energy \citep{oconnellMeasuresDissipationViscoelastic1978} . Intrinsic attenuation is caused by internal dissipation and scattering from crustal and mantle heterogeneities \citep{satoSeismicWavePropagation2009}, which become prominent at high frequencies ($f \geqslant  1$ Hz). As indicated by its nature, $Q$ primarily affects the ground motion amplitudes, but can also prolong the shaking duration, especially for surface waves excited by shallow events \citep{imperatoriRoleTopographyLateral2015, laiShallowBasinStructure2020}. Different tectonic regions are generally characterized by different anelastic attenation regimes. For southern California studied here, a tectonic active area, anelastic attenuation is generally stronger compared to techtonically stable areas \citep{frankelAttenuationHighfrequencyShear1990,ericksonFrequencyDependentLgContinental2004}. The near-surface crustal material experiences more attenuation, where $Q$ can be as low as 10 in soft sediments \citep{asterHighfrequencyBoreholeSeismograms1991,abercrombieNearsurfaceAttenuationSite1997}.

At low frequencies (i.e., less than about 1 Hz), anelastic attenuation is normally modeled as a frequency-independent dependent phenomenon \citep{akiQuantitativeSeismology2002}, but observational evidence shows that $Q$ values appear to increase with frequency at higher frequencies \citeg{akiAttenuationShearwavesLithosphere1980, atkinsonAttenuationSourceParameters1995, ericksonFrequencyDependentLgContinental2004}, at least in some regions. \citet{withersMemoryEfficientSimulation2015} developed an efficient numerical approach to implement the frequency-dependent anelastic attenuation $Q(f)$ in 3D deterministic simulations, and showed that $Q(f)$ models generally predict ground motions better than constant $Q$ models. As simulations are pushed to higher frequencies, anelastic attenuation becomes increasingly important because of the added wave cycles within the simulated domain. An accurate modeling of intrinsic attenuation is critical for accurate estimation of earthquake ground motions.

\subsection{Small-scale Heterogeneities}

The heterogeneous nature of Earth's crust, at different scales, is another important factor governing the propagation of seismic wavefields \citep{levanderSmallscaleHeterogeneityLargescale1992,levanderCrustHeterogeneousOptical1994,beanStatisticalMeasuresCrustal1999,helffrichEarthMantle2001,hedlinSeismicEvidenceSmallscale1997}. Changes in material properties can lead to amplitude decay and dispersive effects, known as scattring. The most prominent effects of scattering on the resulting ground motions is the generation of coda waves, including envelope broadening \citep{satoBroadeningSeismogramEnvelopes1989}, waveform variation and travel time shift \citep{flatteSmallscaleStructureLithosphere1988}. Multiple theoretical studies have been developed to explain the nature of wave scatterings \citep{akiAnalysisSeismicCoda1969, wuMultipleScatteringEnergy1985,akiOriginCodaWaves1975,zengScatteringWaveEnergy1991,zengTheoryScatteredSwave1993,zengSubeventRakeRandom1995}.

Stochastic numerical simulations, typically computationally less demanding than the deterministic counterparts, have been carried out using radiative transfer theory \citep{gusevSimulatedEnvelopesNonisotropically1996,przybillaRadiativeTransferElastic2006} or Markov approximation \citep{saitoEnvelopeBroadeningSpherically2002,sawazakiEnvelopeSynthesisShortperiod2011}. For ground motion predictions and earthquake engineering, hybrid techniques are applied to include scattering statistics, mainly at high frequencies, with parameters constrained by observed seismograms and ground motion prediction models \citep{liuPredictionBroadbandGroundMotion2006,gravesBroadbandGroundMotionSimulation2010, maiHybridBroadbandGroundMotion2010}.
Despite being more computationally expensive, deterministic simulations have been used to study scattering processes, distributions of heterogeneities and scattering characteristics by comparing the synthetic results to seismic records or theoretical predictions \citep{frankelFiniteDifferenceSimulations1986,rothSingleScatteringTheory1993,shapiroSeismicAttenuationScattering1993,thyboSeismicScatteringTop2003}. The potential for extending these studies to higher frequencies, where scattering processes tend to play a larger role, is increasing due to recent availability of high-performance computing resources.

As sufficient coverage of direct measurement constraints for meter-scale mapping of small-scale heterogeneities is prohibitively expensive, statistical methods have been used to parameterize crustal heterogeneities. For example, perturbations of crustal material properties, such as seismic velocities and densities, can be superimposed onto a reference velocity model via a spatial random field, such as the band-limit von Karman correlation function \citep[][\cref{app:A}]{frankelFiniteDifferenceSimulations1986,hartzellEffects3DRandom2010}). Other correlations functions, such Gaussian and exponential formulations, have been less favored for this purpose, as they are unable to match some key scattering phenomena. The parameters that control the distributions of heterogeneities have been constrained by data from [what?]   \citep{thyboSeismicScatteringTop2003,nielsenIdentificationCrustalUpper2006, przybillaEstimationCrustalScattering2009, imperatoriBroadbandNearfieldGround2013, imperatoriRoleTopographyLateral2015}, but remain somewhat uncertain.


\section{Site Amplification}

The amplitude of seismic waves is increased when propagating from stiff bedrock into the lower-velocity soils near the surface due to the impedance constrast \citep{booreShortperiodSwaveRadiation1986,silvaEngineeringCharacterizationStrong1995}. The effects of soft soils on ground motions, referred to as site response, site effects, or site amplification, have been documented and studied for more than 100 years \citeg{wood1908distribution,reid1910california,gutenbergEffectsGroundEarthquake1957}. For example, \citet{gutenbergEffectsGroundEarthquake1957} found that the amplitude of ground motions between 0.67 to 1 Hz were about 5 times larger at dry alluvium sites than at crystalline rock sites.  Different soil types respond differently when excited by ground motions in various frequency bands, dependent on the thickness of the soil column \citep{akiLocalSiteEffects1993}. One of the most outstanding examples of site amplification due to local geological structure was observed during the $M$ 8.1 Michoacan, Mexico, Earthquake in 1985, where the ground motions on soft lake sediments were amplified by to 50 times compared to nearby rock sites \citep{singh1993origin}. Since metropolitan regions are often located on top of soft sediments, the study of local site condition and the resulting amplification is a fundamental goal for earthquake engineering.

The accuracy of site response estimates depends on the accuracy of the subsurface model used, which is usually controlled by the uncertainty of the site properties, and in particular, the shear-wave velocity, $V_S$ \citeg{baraniInfluenceSoilModeling2013,griffithsMappingDispersionMisfit2016}. $V_s$ is generally considered to be the most important parameter for conventional 1D site amplification estimation, which uses an approximation of horizontally polarized plane S waves propagating through a stack of homogeneous, planar layers \citeg{kramerGeotechnicalEarthquakeEngineering1996}. This modeling procedure (SH1D) ignores the lateral complexity of often heterogeneous geology and subsurface structure and is, therefore, not able to capture 2D or 3D amplification effects \citeg{rotenComparisonObservedSimulated2008,thompsonTaxonomySiteResponse2012}. \citet{zhuSeismicAggravationShallow2018} performed numerical analysis on 2D sedimentary basins and found that a constant spectral aggravation factor \citep{chavez-garciaComplexSiteEffects2000}, which quantifies the discrepancy between 1D and 2D/3D models, insufficiently identify basin effects, in particular for regions close to the edge of shallow basins. In addition, ground motions computed in 1D models lack realistic spatial variability, caused by complex wave propagation such as basin amplification, surface-wave generation, and scattering. For this reason, 1D models are unable to capture spatial correlations, which is important for risk calculations, in particular concerning regional-scale infrastructure \citeg{olsenCausesLowfrequencyGround1995,booreCanSiteResponse2004}. Although recent efforts have attempted to reduce velocity uncertainties in site effect estimation \citep{matavosicPracticesProceduresSitespecific2012,teagueMeasuredVsPredicted2018}, these methods either require prohibitively complex processing or are developed for specific cases only.

It is impractical to characterize subsurface structure over a broad region to the resolution required for accurate ground-motion estimation to high frequencies (e.g., 10 Hz, on the order of meters to tens of meters). Instead, some studies choose to use simple proxies, based on broad site classes to supplement estimates of soil properties and spatial site characteristics, for example, the National Earthquake Hazards Reduction Program (NEHRP) soil classification \citep{bssc2003NEHRPRecommended2003,akkarEmpiricalEquationsPrediction2010} or a weighted average of $V_s$ in the uppermost 30 m \citep[$V_{S30}$, e.g., ][]{abrahamsonSummaryAbrahamsonSilva2008,idrissNGAWest2EmpiricalModel2014}. \citet{thompsonTaxonomySiteResponse2012} proposed a scheme to classify surface-downhole site pairs by the extent of interevent variability and goodness of fit between 1D modelling and empirical site response, which can be used to calibrate the constitutive models and guide specific site studies. Despite the use of these characterizations in some generic seismic hazard estimates, for instance, via ground-motion prediction equations, recent work has pointed out the importance of considering site-to-site amplification variability \citep{atkinsonEarthquakeGroundmotionPrediction2006,atikVariabilityGroundmotionPrediction2010}. These studies show that, even within a single NEHRP or $V_{s30}$ class, the variability of site amplification and spatial correlations is sufficiently strong to induce large uncertainty into the resulting ground-motion estimates.

In \cref{chap:etf}, we propose a method to improve estimation of site response estimation and associated uncertainty using surface topography and 3D wave propagation effects using high-resolution 3D numerical simulations. The simulations naturally take advantage of 3D geotechnical information and are able to capture complicated, spatially varying amplification effects.


\renewcommand{\thetable}{\arabic{table}}
\renewcommand{\thefigure}{\arabic{figure}}

\numberwithin{figure}{chapter}
\numberwithin{table}{chapter}
%%%%%%%%
