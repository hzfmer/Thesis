% !TEX encoding = UTF-8 Unicode

\linespread{1.7}
\chapter{Introduction}
\linespread{2.0}
\label{chap:intro}

\section{Motivation}
Earthquake, a globally frequent phenomenon of ground shaking resulting from vast energy release from Earth interior in short time, is one of the most catastrophic natural hazard to human society. While the majority of earthquakes are too small to be felt by humans, they are detected by seismic sensors more than ever and contribute to our understanding of seismology. Yet major earthquakes with magnitude greater than 7 happen more than once per month and still remain not predictable, which necessitates continuing study to mitigate their damage.

Ground motion prediction is central to seismic hazard management, because the ground shaking is oftentimes the most dominant source of damages and immediate cause of following secondary damages, such as tsunamis, landslides, liquefaction and fires. A straightforward approach is to collect earthquake records and perform statistical analysis to formulate empirical functions that incorporate correlations between earthquake characteristics and measurements. This method is referred to as ground motion models (GMMs) or ground motion prediction equations \citep[GMPEs; e.g., ][]{abrahamsonSummaryASK14Ground2014,booreNGAWest2EquationsPredicting2014,campbellNGAWest2GroundMotion2014,chiouUpdateChiouYoungs2014, idrissNGAWest2EmpiricalModel2014}. However, due to sporadicity of extreme earthquakes and sparsity of direct measurements, the estimate of strong ground motions using GMPEs, especially within close distances to earthquakes, suffers from large uncertainty. Numerical methods are therefore widely adopted to investigate both historic and scenario earthquakes, serving as a feasible complement or a substitute to recorded data, in order to estimate seismic rick even beyond the range of existing data. This makes physics-based methods valuable for public earthquake emergency preparedness and design seismic-resistant structures in civil engineering.

Physics-based deterministic earthquake simulation, sometimes also referred to as ground motion prediction, is the process of solving elastodynamic equations numerically by incorporating physical theories with velocity models constrained by direct measurements and indirect inversions. Because analytical methods cannot provide solutions for realistic sophisticated models and stochastic or empirical methods generally provide median values with estimates of uncertainty only, the deterministic numerical methods are developed, including finite-difference, finite-element, boundary element, spectral element and the pseudo-spectral methods \citeg{sanchez1982boundary,frankelThreedimensionalSimulationSeismic1992,olsenThreeDimensionalSimulationMagnitude1995,gravesSimulatingSeismicWave1996,baoLargescaleSimulationElastic1998,seriani3DLargescaleWave1998,kaserArbitraryHighorderDiscontinuous2006, chaljub2006spectral}. These numerical methods differ from each other with individual advantages or disadvantages in terms of accuracy, efficiency and ranges of applicability. Among all these methods, the finite-difference method is still the most widely used in large scale three-dimensional (3D) simulations considering its simplicity in mesh generation and GPU parallelization, efficiency and scalability. Additionally, the finite-difference methods have recently been improved to support plasticity,discontinuous mesh discretization and irregular surface topography \citeg{rotenQuantificationFaultZonePlasticity2017, nieFourthOrderStaggered2017,zhangThreedimensionalElasticWave2012,oreillyHighorderFiniteDifference2021}. Based on extensive verifications and validations, though under various simplifications and assumptions due to insufficiently resolved parameters and computational limits, these methods have been developed and applied successfully in a series of meaningful research practises \citep[e.g., ShakeOut, PetaShake, M9 Cascadia; for more details readers are referred to ][]{cuiPetascaleEarthquakeSimulations2009,cuiTeraShakeComputationalPlatform2009,olsen2009shakeout,marafiImpactsSimulatedM92019}.

With the accelerated advancement of high-performance computing facilities and applications, researchers have achieved considerable progress in higher-frequency ground motion simulations in the last decades \citep{gravesBroadbandSimulationsSouthern2008,olsen2009shakeout,bielakShakeOutEarthquakeScenario2010,roten3DSimulationsEarthquakes2012, savranGroundMotionSimulation2019,withersGroundMotionIntraevent2019}. In the meantime, it is accordingly important to identify different aspects to a finer scale that govern the temporal and spatial characteristics of broadband seismic waves. In general, seismology studies three main aspects: source, path, and site effects, covering a broad range of fields including rupture nucleation, rock failure, plastic deformation, attenuation and scattering, 3D geometrical interference, low-velocity amplification, etc. This thesis will introduce and explore some relevant disciplines in more detail below.


\section{Near-source Plasticity}

A number of densely populated regions are located close to major faults where extreme earthquakes are likely to occur, e.g., Los Angeles, San Francisco, Tokyo, Istanbul and so forth. Nonetheless, near-source ground motions are not sufficiently presented in observed data, which leads to large uncertainty and therefore high ground motion levels at low exceeding probability in probabilistic seismic hazard assessments including building code design and ground motion prediction equations \citeg{hanksObservedGroundMotions2005, bommerWhyModernProbabilistic2006}. Physics-based numerical simulations can complement the GMPEs in these regions where observations are under sampled. Nonetheless, most large-scale simulations consider linear deformation only for modeling simplicity \citep{olsen20083d,molnar2014earthquake}, while studies have shown that crust rocks in the fault damage zone and near the surface experience plastic yielding in extreme earthquakes, which limits the peak ground motions, especially in the near field \citep{andrewsPhysicalLimitsGround2007,ma2008physical,duanSensitivityStudyPhysical2010,templetonDynamicRuptureBranched2010,dunhamEarthquakeRupturesStrongly2011}. \citet{fieldNonlinearSiteResponse1998} reviewed laboratory results and observations and claimed non-linearity is most effective at frequencies above 1 Hz. \citet{gravesBroadbandSimulationsSouthern2008} found that non-linear material response can reduce the peak amplitudes of long-period (<1 s) ground motions by up to 70\% in the Los Angeles Basin compared to the case with visco-elastic models in numerical simulations.

% More to be added from the EKS draft
%The numerical simulations provide an viable approach to establish physical limits in predicting ground motions in such regions. softening and damping


\section{Model characteristics in High-Frequency Simulations}

Seismic velocity models that incorporate geological, geotechnical and geophysical in formations are vastly used in forward simulations. The accuracy of these models are important in minimizing the systematic uncertainties in the following modeling procedures and resulting reliable ground motion predictions. In the past decades, most large-scale deterministic earthquake simulations have been focused on low-frequency ($f_{max}$ $\leqslant$ 1 Hz) seismic waves, which allows rough quantification of model properties due to the intrinsic low resolution in the simulation approach. Generally, these simulations excludes surface topography, small-scale spatial heterogeneities and frequency-dependent attenuation, along with the minimum velocity clamped at relatively high values to alleviate computational costs. However, these characteristics are expected to matter much more when entering the high-frequency bandwidth and, more importantly coupled with each other to impose complicated wavefield modification. For example, \citet{laiShallowBasinStructure2020} found that surface waves triggered by shallow sources are scattered by both topography and velocity heterogeneities that dominate synthetics over a wide range distance range, and weaker attenuation may increase the envelope duration.
The sections below briefly discuss the aforementioned effects according to previous studies and how they can be accommodated and addressed in high-frequency deterministic simulations.


\subsection{Near surface Low Velocity}
The presence of near-surface low-velocity layers has a dramatic effect on the simulated ground motion \citep{imperatoriRoleTopographyLateral2015}.
\citet{shawUnifiedStructuralRepresentation2015} investigated the effects of different velocity models and found significant sensitivity of ground motions to the velocity model. Previous studies have evidenced that low near-surface material properties, $V_S$ in particular, can lead to significant amplification, especially at higher frequencies, e.g. 0.5-10 Hz \citep{booreSiteAmplificationsGeneric1997,poggiDerivationReferenceShearWave2011}. \citet{imperatoriBroadbandNearfieldGround2013} reported that low-velocity layers increase coda wave amplitude and duration by trapping energy close to the free surface.
Most large-scale deterministic simulations, however, adopt a larger than realistic surface low velocities, because of the limitation of modeling methods and computing capacity, as well as the common understanding that current crustal model of the Earth is insufficient to accurately model demonstrate the complex scattering the interference in soft shallow soils.

The quantitative identification of properties of shallow materials is one of the main tasks for earthquake modellers. This is not a problem for local site study, because numerous well-established techniques have been applied for this purpose, including seismic borehole logging, surface-wave dispersion survey, cone penetration test, gravity observations and oil-well samples. The regional-scale measurements, however, are technically infeasible and financially prohibitive. For this reason, researchers always rely on 3D crustal velocity model, along with empirical functions, to address seismic hazard assessments. The challenge is then raised improve the accuracy and resolution of shallow crustal materials.

Our method (see \cref{chap:vs30}) is an update of the \citet{elyVs30derivedNearsurfaceSeismic2010} method that enhances the shallow velocities, fulfilling the need for detailed description of near-surface structure in numerical simulations, constrained from both the very uppermost (~30 m) geological and geotechnical information (e.g., $V_{S30}$) and the deeper tomographic inversions \citeg{leeTestingWaveformPredictions2014,shawUnifiedStructuralRepresentation2015}.

%Different SCEC CVMs, e.g., CVM-S4.26M01 and CVM-H15.1, all using tomographic and source inversions as base models with basin information incorporated, show similarly poor resolution in mountainous areas where geological measurements are sparse due to prohibitive cost. We expect our proposed shallow velocity refinement to achieve similar improvements in ground motion predictions at these rock stations.


\subsection{Topography}
Surface topography is known to induce severer seismic risk by amplifying ground motions at high elevations, compared to flat grounds or hollow topographies \citep{celebiTopographicalGeologicalAmplifications1987,kawaseTopographyEffectCritical1990,massaExperimentalApproachEstimating2010,burjanekEmpiricalEvidenceLocal2014}. The key point in exploring topographic effects is to isolate the contribution of local relief from other factors, such as stratigraphy, the presence of fault damage zone or low-velocity surface layers. In addition, the ideal reference point, which is supposed to share similar geology with topographic areas while exempt from topographic effects \citep{celebiTopographicalGeologicalAmplifications1987,geliEffectTopographyEarthquake1988,chavez-garciaComplexSiteEffects2000}, is difficult to determine in reality. Numerical simulations are therefore applied to overcome such limitations and produce flexible ranges of characterizing topographic amplification \citep{booreNoteEffectSimple1972,sanchez-sesmaDiffractionSVRayleigh1991,lovati2011estimation,hartzellGroundMotionPresence2017}.

The topographic regions are affected in a frequency band,which is correlated with the characteristic dimensions of the relief. In addition, a wide range geomorphometric parameters are used to describe topography geometry, including slope, curvature, relative elevation, surface roughness are investigated \citep{ashfordAnalysisTopographicAmplification1997,nguyenEvaluationSeismicGround2007,bouckovalasNumericalEvaluationSlope2005}. It is found that smoothed curvature and relative elevation, which are linearly correlated with each other, are mostly relevant in topographic amplification \citep{maufroyFrequencyScaledCurvature2015,raiEmpiricalTerrainBasedTopographic2017}. Because the characteristic length is critical and dependent on frequency, the topographic effects behave differently in different frequency bands. As the frequency increases, the conventional assumptions that topography has negligible effects below 1 Hz \citep{booreNoteEffectSimple1972, pischiuttaTopographicEffectsHill2010}, may not hold and thus is needed in high-frequency simulations.


\subsection{Anelastic attenuation}

Anelastic attenuation of seismic waves during propagation thorough earth crust can be described as a fractal energy loss per cycle: $Q^{-1}(\textbf{x}, f) = \sigma E / 2\pi E$, where $\textbf{x}$ is the position and $f$ is the circular frequency and $E$ is the total energy \citep{oconnellMeasuresDissipationViscoelastic1978}. The attenuation, as the inverse of the quality factor $Q$, is composed of internal dissipation and scattering from lithosphere heterogeneities \citep{satoSeismicWavePropagation2009}, which become prominent at high frequencies ($f \geqslant  1$ Hz). As indicated by its nature, $Q$ affects the ground motion amplitudes mainly, though studies also show that weaker attenuation may prolong the shaking duration, especially on surface waves incited by shallow events \citep{imperatoriRoleTopographyLateral2015, laiShallowBasinStructure2020}. Particularly, in southern California, which is the region to study in this chapter, the anelastic attenuation is assumed to have stronger effects due to the lower $Q$ values for both compressional ($P$) and shear ($S$) waves compared to techtonically stable areas \citep{ericksonFrequencyDependentLgContinental2004,frankelAttenuationHighfrequencyShear1990}. In addition, the near-surface crust experiences more attenuation, where $Q$ can be as low as 10 in soft sediments \citep{abercrombieNearsurfaceAttenuationSite1997,asterHighfrequencyBoreholeSeismograms1991}.

$Q$ is normally modeled as frequency-dependent at low frequencies \citep{akiQuantitativeSeismology2002}, but observed to increase with frequency above 1 Hz at a power law rate between 0 to 1 \citeg{akiAttenuationShearwavesLithosphere1980, atkinsonAttenuationSourceParameters1995, ericksonFrequencyDependentLgContinental2004}. \citet{withersMemoryEfficientSimulation2015} applied the memory-variable approach \citep{day1984numerical,emmerich1987incorporation,blanch1995modeling} to implement the frequency-dependent $Q$ in a 3D staggered-grid finite difference scheme, and showed that $Q(f)$ models generally predict ground motions better than constant $Q$ models. As the simulations are pushed to higher frequencies, the anelastic attenuation becomes progressively important because of the increasing wave cycles within the simulated domain. An accurate modeling of attenuation, along with the power-law exponent in accordance, is critical in physics-based ground motion simulations.

\subsection{Small-scale Heterogeneities}

The heterogeneous nature of the Earth crust, at different scales, is one of the most important factors governing the propagation of seismic wavefields \citep{levanderSmallscaleHeterogeneityLargescale1992,levanderCrustHeterogeneousOptical1994,beanStatisticalMeasuresCrustal1999,helffrichEarthMantle2001,hedlinSeismicEvidenceSmallscale1997}. The most prominent phenomenon due to scattering is relevant with coda waves, including envelope broadening \citep{satoBroadeningSeismogramEnvelopes1989}, waveform variation and travel time shift \citep{flatteSmallscaleStructureLithosphere1988}. Multiple theoretical studies have been developed to explain the nature of wave scatterings \citep{akiAnalysisSeismicCoda1969, wuMultipleScatteringEnergy1985,akiOriginCodaWaves1975}, and eventually condensed into the multiple shear-to-shear backscattering theory \citep{zengScatteringWaveEnergy1991,zengTheoryScatteredSwave1993,zengSubeventRakeRandom1995}.

Deterministic numerical simulations are also extensively used to study the scattering process, distributions of heterogeneities and scattering characteristics by comparing the synthetic results to data records or theretical predictions \citep{frankelFiniteDifferenceSimulations1986,rothSingleScatteringTheory1993,shapiroSeismicAttenuationScattering1993,thyboSeismicScatteringTop2003}. Not until the last decade, regional-scale elastic 3D simulations are carried out for frequency above 1 Hz due to the computational expense \citep{hartzellEffects3DRandom2010,pitarka2009simulating}. On the other hand, stochastic numerical simulations are performed using radiative transfer equation \citep{gusevSimulatedEnvelopesNonisotropically1996,przybillaRadiativeTransferElastic2006} or Markov approximation \citep{saitoEnvelopeBroadeningSpherically2002,sawazakiEnvelopeSynthesisShortperiod2011}. For the sake of ground motion predictions and earthquake engineering, hybrid techniques are applied to emulate scattering statistics, mainly at high frequencies, in order to match observed seismograms and ground motion prediction models \citep{liuPredictionBroadbandGroundMotion2006,gravesBroadbandGroundMotionSimulation2010, maiHybridBroadbandGroundMotion2010}. The lack of considering real medium heterogeneities, however, precludes the presence of spatial correlation within short distance between sites.

The perturbation of crustal material properties is generally superimposed onto the base deterministic velocity model via a spatial random field, such as the band-limit von Karman correlation function \citep[][\cref{app:A}]{frankelFiniteDifferenceSimulations1986,hartzellEffects3DRandom2010}), along with Gaussian and exponential correlation functions, which are less favorable for being unable to match some key scattering phenomena. The determination of the spatial parameters that control the heterogeneities remain largely uncertain yet \citeg{dolanRemarksEstimationFractal1997}, though researchers have been narrowing the uncertainty \citep{thyboSeismicScatteringTop2003,nielsenIdentificationCrustalUpper2006, przybillaEstimationCrustalScattering2009, imperatoriBroadbandNearfieldGround2013, imperatoriRoleTopographyLateral2015}


\section{Site Amplification Due To 3D Structure}

The amplitude of seismic waves is increased when propagating from stiff bedrocks into the lower-velocity soils near the surface \citep{booreShortperiodSwaveRadiation1986,silvaEngineeringCharacterizationStrong1995}. The effects of soft soils on ground motions, referred to as site response (or site effects, site amplification), have been documented and studied since as early as 1900s \citeg{wood1908distribution,reid1910california,gutenbergEffectsGroundEarthquake1957}. For example, \citet{gutenbergEffectsGroundEarthquake1957} found that the amplitude of ground motions between 0.67 to 1 Hz were 5 times larger at dry alluvium sites than at crystalline rock sites.  Different soil types respond differently when incited by ground motions with various frequency bands, dependent on the depth of underlying soil column \citep{akiLocalSiteEffects1993}. One of the most outstanding example of site amplification due to local geology structure was observed during the $M$ 8.1 Mexico Michoacan Earthquake in 1985, where the ground motions on soft lake sediments were amplified by a factor of up to 50 compared to nearby competent sites \citep{singh1993origin}. In view of the concern about the potential hazard over metropolitan regions underlain by soft and young sediments, the study of local site condition and the resulting amplification is a fundamental goal for earthquake engineering.

The accuracy of site response estimates depends on the accuracy of the subsurface model used, and this is usually assumed to be controlled by the uncertainty in the site properties, in particular, the shear-wave velocity, $V_S$ \citeg{baraniInfluenceSoilModeling2013,griffithsMappingDispersionMisfit2016}. $V_s$ is the most important parameter for conventional 1D modeling of the TF, in which it is assumed that surface (and subsurface) motion consists of horizontally polarized plane S waves propagating through a stack of homogeneous layers \citeg{kramerGeotechnicalEarthquakeEngineering1996}. This modeling procedure (SH1D) ignores the lateral complexity of the often heterogeneous geology and subsurface structure and is, therefore, not able to include potential 2D and 3D amplification effects in the observations \citeg{rotenComparisonObservedSimulated2008,thompsonTaxonomySiteResponse2012}. \citet{zhuSeismicAggravationShallow2018} performed numerical analysis on 2D basins and found that a constant spectral aggravation factor \citep{chavez-garciaComplexSiteEffects2000}, which quantifies the discrepancy between 1D and 2D/3D models, is insufficient to identify basin effects, especially, in close-to-edge regions of shallow basins. Both observations and analytical solutions suggest that 1D models lack an estimate of spatial variability, caused by complex wave propagation such as basin amplification, surface-wave generation, and scattering, and are, therefore, unable to capture spatial correlations, which may be important for understanding risk, especially, to regional-scale infrastructure \citeg{olsenCausesLowfrequencyGround1995,booreCanSiteResponse2004}. Although, recent approaches have attempted to reduce velocity uncertainties in site effect estimation \citep{matavosicPracticesProceduresSitespecific2012,teagueMeasuredVsPredicted2018}, these methods either require prohibitively complex processing or are developed for specific cases only.

It is impractical to constrain subsurface structure over a wide region to the resolution (on the order of meters to tens of meters) required for accurate ground-motion estimation to high frequencies (e.g., 10 Hz). Instead, some studies choose to use simple proxies, based on broad site classes to supplement estimates of soil properties and site spatial characteristics, for example, the National Earthquake Hazards Reduction Program (NEHRP) soil classification \citep{bssc2003NEHRPRecommended2003,akkarEmpiricalEquationsPrediction2010} or a weighted average of $V_S$ in the uppermost 30 m \citep[$V_{S30}$, e.g., ][]{abrahamsonSummaryAbrahamsonSilva2008,idrissNGAWest2EmpiricalModel2014}. \citet{thompsonTaxonomySiteResponse2012} proposed a scheme to classify surface-downhole site pairs by the extent of interevent variability and goodness of fit between 1D modelling and empirical site response, which can be used to calibrate the constitutive models and guide specific site studies. Despite the use of these characterizations in some generic seismic hazard estimates, for instance, via ground-motion prediction equations, recent work has pointed out the importance of considering site-to-site amplification variability \citep{atkinsonEarthquakeGroundmotionPrediction2006,atikVariabilityGroundmotionPrediction2010}. These studies show that, even within a single NEHRP or $V_{S30}$ class, the variability of site amplification and spatial correlations is strong enough to contribute significant uncertainty in ground-motion estimates.

In \cref{chap:etf}, we propose a method to constrain the near-surface properties using surface topography and perform high-resolution 3D numerical simulations to investigate the uncertainty in site response modeling. The simulations naturally take advantage of 3D geotechnical information and are able to incorporate complicated spatially varying amplification effects.


\renewcommand{\thetable}{\arabic{table}}
\renewcommand{\thefigure}{\arabic{figure}}

\numberwithin{figure}{chapter}
\numberwithin{table}{chapter}
%%%%%%%%
