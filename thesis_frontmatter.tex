%
%
% UCSD Doctoral Dissertation Template
% -----------------------------------
% http://ucsd-thesis.googlecode.com
%
%


%% REQUIRED FIELDS -- Replace with the values appropriate to you

% No symbols, formulas, superscripts, or Greek letters are allowed
% in your title.
\title{Toward high-frequency determinstic simulations: source, path and site effects}

\author{Zhifeng Hu}
\degreeyear{\the\year}

% Master's Degree theses will NOT be formatted properly with this file.
\degreetitle{Doctor of Philosophy}

\field{Geophysics}
%\specialization{Anthropogeny}  % If you have a specialization, add it here

\chair{Professor Kim Olsen}
% Uncomment the next line iff you have a Co-Chair
%\cochair{Professor }
%
%
% Or, uncomment the next line iff you have two equal Co-Chairs.
%\cochairs{Professor Chair Masterish}{Professor Chair Masterish}

%  The rest of the committee members  must be alphabetized by last name.
\othermembersucsd{
    \hspace*{0.5in} Professor Joel Conte\\
    \hspace*{0.5in} Professor Peter Shearer\\
}

\othermemberssdsu{
    \hspace*{0.5in} Professor Steven Day\\
    \hspace*{0.5in} Professor Samuel Shen\\
    \hspace*{0.5in} Doctor Daniel Roten
}
\numberofmembers{6} % |chair| + |cochair| + |othermembers|
\urlstyle{same}

%% START THE FRONTMATTER
%
\begin{frontmatter}

    %% TITLE PAGES
    %
    %  This command generates the title, copyright, and signature pages.
    %
    \makefrontmatter

    %% DEDICATION
    %
    %  You have three choices here:
    %    1. Use the ``dedication'' environment.
    %       Put in the text you want, and everything will be formated for
    %       you. You'll get a perfectly respectable dedication page.
    %
    %
    %    2. Use the ``mydedication'' environment.  If you don't like the
    %       formatting of option 1, use this environment and format things
    %       however you wish.
    %
    %    3. If you don't want a dedication, it's not required.
    %
    %
    \begin{dedication}
        \vspace*{\fill}
        To my family \par Xiaoyang, Xiuhong and Fei \par
        \vspace*{\fill}
    \end{dedication}


    % \begin{mydedication} % You are responsible for formatting here.
    %   \vspace{1in}
    %   \begin{flushleft}
    % 	To me.
    %   \end{flushleft}
    %
    %   \vspace{2in}
    %   \begin{center}
    % 	And you.
    %   \end{center}
    %
    %   \vspace{2in}
    %   \begin{flushright}
    % 	Which equals us.
    %   \end{flushright}
    % \end{mydedication}



    %% EPIGRAPH
    %
    %  The same choices that applied to the dedication apply here.
    %
    \begin{epigraph} % The style file will position the text for you.

        \emph{
            You must know that a person’s ability to discern the truth \\
            is directly proportional to his knowledge.
        }\\
        --- Cixin Liu, The Three-Body Problem \par

        \setlength{\parskip}{40pt plus 1pt minus 1pt}
        \emph{There are only the pursued, the pursuing, the busy and the tired.}\\
        --- F. Scott Fitzgerald, The Great Gatsby \par

        \setlength{\parskip}{40pt plus 1pt minus 1pt}
        \emph{Never confuse education with intelligence, \\
            you can have a PhD and still be an idiot.}\\
        --- Richard P. Feynman \par


    \end{epigraph}

    % \begin{myepigraph} % You position the text yourself.
    %   \vfil
    %   \begin{center}
    %     {\bf Think! It ain't illegal yet.}
    %
    % 	\emph{---George Clinton}
    %   \end{center}
    % \end{myepigraph}


    %% SETUP THE TABLE OF CONTENTS
    %
    \tableofcontents
    \listoffigures  % Comment if you don't have any figures
    \listoftables   % Comment if you don't have any tables



    %% ACKNOWLEDGEMENTS
    %
    %  While technically optional, you probably have someone to thank.
    %  Also, a paragraph acknowledging all coauthors and publishers (if
    %  you have any) is required in the acknowledgements page and as the
    %  last paragraph of text at the end of each respective chapter. See
    %  the OGS Formatting Manual for more information.
    %
    \begin{acknowledgements}

        % I have always been recalling the time I walked along the La Jolla beach years ago, when the afterglow of sunset lingered forever, and the path ahead extended beyond the sight of my eyes. It came to me that the time paused, and I could repeat those days once and once again, which, I do not know when, fades out and now the end of this journey arrives. Fortunately, I have been accompanied by the greatest people, to whom I owe my sincere appreciation.

        First and foremost, I would like to express my deepest sense of gratitude to my supervisor, Professor Kim Olsen, for his invaluable supervision, support, and tutelage during the course of my PhD studies. Kim always gives me the freedom to explore problems, allows me to try different approaches, and re-direct me every time I move in a wrong direction. He is always responsive to asking and patient to answering, even if my questions are naive. His suggestions and comments are so detailed and constructive that they easily refresh my thoughts and guide me in the right direction. I am always impressed by his skills in applying physical sense in solving problems and providing relevant knowledge in a timely manner. He emphasizes the big picture in seismology, while at the same time taught me how to present scientific findings, polish academic reports and communicate with the community. I appreciate the wealth of knowledge he has given me.

        It is my privilege to thank my committee members for their great mentorship and guidance. I thank Professor Steve Day, who is always supportive and encouraging. Even during the most difficult times, I feel encouraged thinking of his optimistic attitude and enthusiasm in life. He is rigorous to scientific problems but open to explore different possibilities, to which he always provides valuable comments and insights. It was an honor to attend his retirement ceremony in person and see how a life of contribution to science is remembered and respected. Professor Peter Shearer is the person leading me to this field by his popular course "Introduction to Seismology", which is still an important source of answers in my research. He showed me that science can be elegant yet fun. I thank Professor Samuel Shen for his valuable help in mathematics in our discussions. I also thank Professor Joel Conte for his kind suggestions and comments. Dr. Daniel Roten is extraordinarily supportive in Kim's group, who taught me the details in research practice from the beginning. I cannot overstate how much I learned from him, and it is my privilege to work with someone as versatile, brilliant, and productive as Daniel. I am also grateful to Professor Shuo Ma at SDSU for sharing his experience and life advice.

        I am grateful to all the professors at Scripps Institution of Oceanography (SIO) and SDSU, who taught a wide series of courses that prepared me for my research, including Professor Duncan Agnew, Professor Adrian Borsa, Professor Catherine Constable, Professor Steven Constable, Professor Yuri Fialko, Professor Guy Masters, Professor David Sandwell, Professor Peter Shearer, and Professor Dave Stegman at Scripps Institution of Oceanography at UCSD; Professor Michael Holst in the Department of Mathematics at UCSD; and Professor Shuo Ma and Professor Kim Olsen at SDSU.

        My research achievements would not be possible without the help and communication from my coauthors and collaborators from the Southern California Earthquake Center. I thank Kim Olsen, Steven Day, Daniel Roten, Christine Goulet, Robert Graves, Fabio Silva, Scott Callaghan, and Kevin Milner for their incredible collaborations and discussions.

        I appreciate the support and friendship from my incredible fellows and friends: Nan Wang, Te-Yang Yeh, Yongfei Wang, Shiying Nie, Qian Yao, Wei Wang, Zefeng Li, Yuxiang Zhang, Bill Savran, Kyle Withers, Drake Singleton, Rumi Takedatsu, Xiaohua Xu, Kang Wang, Wenyuan Fan, Zhao Chen, Yuval Levy, Zeyu Jin, Yue Du, Zheng Fang, Junlong Hou, Mingda Li, and Yuan Huang. Thank you for making my time at San Diego a wonderful memory. I also appreciate the administrative advisors and staff, including Gilbert Bretado, Wayne Farquharson, Paul Dean, and Sara Miceli at UCSD; Irene Occhiello, Pia Parrish, Heather Webb, Susan Langsford, and Pat Walls at SDSU, for their consistent and timely assistance.

        Finally, I would also like to extend my deepest gratitude to my parents, Xiaoyang and Xiuhong, for always listening to me, supporting me, and encouraging me. Lastly but most importantly, I thank Fei for being there for me, for always believing in me, and for enriching my life in numerous amounts of ways.

        \bigskip

        
        \Cref{chap:vs30}, in full, is a reformatted version of a paper currently being prepared for submission for publication: Hu, Z., Olsen, K.B., and Day S.M (2021). Calibration of the Near-surface Seismic Structure in the SCEC Community Velocity Model Version 4.
        The dissertation author was the primary investigator and author of this paper.
        \Cref{chap:highf}, in full, is a reformatted version of a paper currently in preparation for submission for publication: Hu, Z. and Olsen, K.B. (2021), 0-5 Hz Deterministic 3D Ground Motion Simulations for the 2014 Mw 5.1 LaHabra Earthquake. The dissertation author was the primary investigator and author of this paper.
        \Cref{chap:etf}, in full, is a reformatted version of the material as it appears in the Bulletin of the Seismological Society of America: Hu, Z., Roten, D., Olsen, K. B, and Day, S.M. (2020). Modeling of Empirical Transfer Functions with 3D Velocity Structure. \emph{Bulletin of the Seismological Society of America}.
        The dissertation author was the primary investigator and author of this paper.
        \Cref{chap:eks}, in full, is a reformatted version of a paper currently in preparation for submission for publication: Hu, Z., Roten, D., Olsen, K.B., and Day, S.M. (2021). Kinematic Source Models for Earthquake Simulations with Fault-zone Plasticity. The dissertation author was the primary investigator and author of this manuscript.
        

    \end{acknowledgements}


    %% VITA
    %
    %  A brief vita is required in a doctoral thesis. See the OGS
    %  Formatting Manual for more information.
    %
    \begin{vitapage}
        \begin{vita}
            \item[2015] B.S. in Geophysics, Peking University
            \item[2015-2021] Graduate Teaching/Research Assistant, San Diego State University
        \end{vita}
        \begin{publications}
            \item \textbf{Hu, Z.}, Roten, D., Olsen, K.B., and Day, S.M. (2021) Kinematic Source Models for Earthquake Simulations with Fault-zone Plasticity. \emph{In preparation for publication}.
            \item \textbf{Hu, Z.} and Olsen, K.B. (2021). 0-5 Hz Deterministic 3D Ground Motion Simulations for the 2014 Mw 5.1 LaHabra Earthquake. \emph{To be submitted for publication}.
            \item \textbf{Hu, Z.}, Olsen, K.B., and Day, S.M. (2021). Calibration of the Near-surface Seismic Structure in the SCEC Community Velocity Model Version 4. \emph{To be submitted for publication}.
            \item O'Reilly O., Yeh, T. Olsen, K.B., \textbf{Hu, Z.}, Breuer, A.N., Roten D., Goulet C. (2021). A High-order Finite Difference Method on Staggered Curvilinear Grids for Seismic Wave Propagation Applications with Topography, \emph{Bulletin of the Seismological Society of America}. \emph{Accepted for publication, June 10, 2021}.
            \item \textbf{Hu, Z.}, Roten, D., Olsen, K.B., and Day, S.M. (2020). Modeling of Empirical Transfer Functions with 3D Velocity Structure. \emph{Bulletin of the Seismological Society of America}. \url{doi:10.1785/0120200214}.


        \end{publications}
    \end{vitapage}


    %% ABSTRACT
    %
    %  Doctoral dissertation abstracts should not exceed 350 words.
    %   The abstract may continue to a second page if necessary.
    %
    \begin{abstract}
        High-frequency (here defined as $f_{max}$ > 1 Hz) ground motions are closely relevant to building response associated with small structures of engineering interests. Gaining a deeper understanding of the propagation of figh-frequency seismic waves and characteristics of the resulting ground motions is therefore a principal goal for seismologists and earthquake engineers. Earthquake simulations, in particular those based on physics-based and deterministic models, have drawn significant attention from the seismic community in the most recent decades as a valuable supplement to recorded data. With the potential ability to accurately characterize broadband wavefields, numerical simulations have their own limitations, in particular difficulty in characterizing the underlying physical parameters to a sufficient resolution and computationally accommodating regional-scale models for seismic hazard analysis. The primary objective of this dissertations is to explore and rate the effects on high-frequency ground motions from various model features. Chapter \Cref{chap:intro} is an introduction, providing background and motivation for each of the following chapters. \Cref{chap:eks} studies nonlinear effects using dynamic simulations and proposes an equivalent kinematic source generator to emulate near-source plasticity in terms of the resulting peak ground velocities. Chapters \hyperref[chap:vs30]{3} and \hyperref[chap:highf]{4} focus on model characteristics that govern the high-frequency ground motion. \Cref{chap:vs30} proposes a calibration approach that enhances the near-surface velocity structure, that is insufficiently resolved in community velocity models. In \Cref{chap:highf}, we simulate a series of models to investigate the effects of topography, small-scale heterogeneities, frequency-dependent attenuation, and low near-surface velocities on the resulting wavefields and ground motions as the frequency extends up to 5 Hz. In \Cref{chap:etf}, we propose a method to incorporate surface topography in constraining the 3D subsurface structure to predict site response.

        % High-frequency (> 1 Hz) ground motions are closely relevant to building response associated with small structures of the engineering interests. Gaining an deeper understanding of the propagation of seismic waves and characteristic of ground motions, is therefore a principal goal for seismologists and earthquake engineers. Earthquake simulations, physcis-based deterministic simulations in particular, as a valuable complement to (often inadequate) recorded data, have drawn significant attention from the seismic community in the last decades. With the potential ability to accurately characterize broadband wavefield, numerical simulations have their own limitations, namely the difficulty in charaterizing the underlying physical parameters in fine scale and accommodating regional-scale domains for risking earthquake study. The primary objective of this dissertations is to explore various model properties that impose high-frequency effects. Chapter 1 is an introduction, providing background and motivation for each of the following chapters. Chapter 2 Cref{chap:eks} studies nonlinear effects using dynamic simulations and propose an equivalent kinematic source generator to emulate near-source plasticity in terms of the resulting peak ground velocities. In Chapter 3 Cref{chap:etf}, we incoporates surface topography in constraining the 3D subsurface structure to predicte site response. Chapter 4 \& 5 Cref{chap: vs30, chap: highf} focus on model features that determine the high-frequency ground shaking. Chapter 4 Cref{chap:vs30} proposes a calibration approach that refines the near-surface velocity structure insufficiently resolved in community velocity models. In Chapter 5 Cref{chap:highf}, we intensively simulate a series of models with topography, small-scale heterogeneities, frequency-dependent attenuation, low near-surface velocities to investigate their contributions in modulating wavefields and ground motions as the frequency extends up to 5 Hz.

    \end{abstract}


\end{frontmatter}
